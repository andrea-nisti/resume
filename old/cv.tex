%%% CV TEMPLATE 

\documentclass[10pt,a4paper,sans]{moderncv}        % possible options include font size ('10pt', '11pt' and '12pt'), paper size ('a4paper', 'letterpaper', 'a5paper', 'legalpaper', 'executivepaper' and 'landscape') and font family ('sans' and 'roman')

% modern themes
\moderncvstyle{banking}                            % 'casual' (default), 'classic', 'oldstyle' and 'banking'
\moderncvcolor{green}                               % 'blue, 'orange', 'green', 'red', 'purple', 'grey', 'black'
\usepackage[utf8]{inputenc}
\usepackage{lastpage}
\rfoot{\addressfont\itshape\textcolor{gray}{\thepage\ / \pageref{LastPage}}}

% adjust the page margins
\usepackage[scale=0.75]{geometry}
%\setlength{\hintscolumnwidth}{3cm}               
% \setlength{\maketitlenamewidth}{10cm}          

\usepackage{import}
\usepackage[absolute,overlay]{textpos} 

% set font to Helvetica <3<3<3
\usepackage[scaled]{helvet}
\renewcommand\familydefault{\sfdefault} 
\usepackage[T1]{fontenc}

% personal data
\name{Andrea}{Nisticò}                               
\address{Via Macaggi, 18/5}{16121 Genova}{Italy}
\phone[mobile]{+39 320 1108745}                   
\social[github]{andrea-nisti}
\email{	andrea.nistic@gmail.com}
%% Stuff for the references
\newcommand{\cvdoublecolumn}[2]{               % Command for double column references
  \cvitem[0.75em]{}{%
    \begin{minipage}[t]{0.5\textwidth}#1\end{minipage}%
    \hfill%
    \begin{minipage}[t]{0.5\textwidth}#2\end{minipage}%
    }%
}

\newcommand{\cvreference}[7]{
    \textbf{#1}\newline                                           % Name
    \ifthenelse{\equal{#2}{}}{}{\addresssymbol~#2\newline}        % {Department}
    \ifthenelse{\equal{#3}{}}{}{#3\newline}%                      % {University}
    \ifthenelse{\equal{#4}{}}{}{#4\newline}%                      % {address line 1}
    \ifthenelse{\equal{#5}{}}{}{#5\newline}%                      % {address line 2}
    \ifthenelse{\equal{#6}{}}{}{\emailsymbol~\texttt{#6}\newline} % {e-mail address}
    \ifthenelse{\equal{#7}{}}{}{\phonesymbol~#7}}                 % {phone number}

%----------------------------------------------------------------------------------
%            content
%----------------------------------------------------------------------------------

\begin{document}
\nopagenumbers{}
%\begin{CJK*}{UTF8}{gbsn}                          % to typeset your resume in Chinese using CJK
%-----       resume       ---------------------------------------------------------

\makecvtitle
\small{\textbf{About me.} I am a young and motivated roboticist with a strong interest in software architectures, applied control and an international background. I love challenges and learning from the others, I maturated a deep interest in MAVs (Micro Aerial Vehicles) and aerial technology during my master thesis. } 

\vspace{5pt}

\section{Education}
\subsection{Academic Qualifications}
\vspace{8pt}

\begin{itemize}

	\item{\cventry{2014--2015}{MEng EMARO European Master on Advanced Robotics}{Università di Genova}{Genoa, Italy}{\textit{104/110}}{Double Degree Program (II Year)}{First in score ranking}}
   
	\vspace{3pt}
	
	\item{\cventry{2013--2014}{MEng EMARO European Master on Advanced Robotics}{École Centrale de Nantes}{Nantes, France}{}{Double Degree Program (I Year)}}
	
	\vspace{3pt}
	
	\item{\cventry{2010--2013}{BEng Engineering Sciences }{Università di Roma Tor Vergata}{Rome, Italy}{\textit{110/110 Cum Laude}}{Mechatronics background, strong basis in physics and mathematics}}

\end{itemize}
\subsection{Extras}
\begin{itemize}
    \item{\cventry{December 2020}{FreeRTOS Real-Time Programming}{Doulos Online course }{doulos.com}{}{Theory and exercises on Real-Time concepts, development of real time applications with FreeRTOS}}

     \item{\cventry{October 2018}{Ethereum developer certification}{B9Lab Online course }{b9lab.com}{}{Blockchain theory, development of decentralised applications on the Ethereum platform}}
     
     \item{\cventry{October 2018}{HyperLedger Fabric developer certification}{B9Lab Online course}{b9lab.com}{}{Blockchain theory, how HyperLedger works and hands-on projects with HLF framework}}

    \item{\cventry{Summer 2017}{RegML PhD course}{Università di Genova}{Genoa, Italy}{}{Theory and exercises on regularization methods for machine learning}}

	\item{\cventry{Summer 2016}{TRADR/EuRathlon Summer School on heterogeneity in robotics}{University of Oulu }{Oulu, Finland}{}{Theory and exercises on different robotics systems (UAVs and ground vehicles)}}
    
    \item{\cventry{Summer 2015}{TRADR Summer School on Autonomous Micro Aerial Vehicles}{Fraunhofer institute}{Bonn, Germany}{}{}}

\end{itemize}
\vspace{5pt}

\section{Employment}

\vspace{8pt}
\begin{itemize}

\item{\cventry{June 2018--Now}{Researcher and Software Developer}{Teseo Srl}{Genoa, Italy}{}{Development of ambient intelligence solutions for the Kibi project: indoor localization for elderly care applications on embedded wearable devices}}
\vspace{6pt}

\item{\cventry{September 2017--Now}{Teaching assistant}{University of Genova}{Genoa, Italy}{}{Teaching and conducting lab sessions on Robot Programming to master students. Main topics are: ROS programming and GazeboSim}}
\vspace{6pt}

\item{\cventry{February 2016--February 2018}{Research Fellow }{University of Genova }{Genoa, Italy}{}{Development of a framework for autonomous navigation, target location and landing on a moving target for UAVs}}
\vspace{6pt}
\item{\cventry{Summer 2014}{Intern at iCub Facility}{Italian Institute of Technology}{Genoa, Italy}{}{Investigation of a possible new middleware protocol for the iCub robot based on DDS protocol}}

\item{\cventry{September 2012}{EFMC9 Staff}{University of Roma Tor Vergata}{Rome, Italy}{}{Worked as part of the organizing staff at the 9th European Fluid Mechanics Conference}}
\end{itemize}

\vspace{5pt}

\section{Relevant Experience}

\vspace{8pt}

\begin{itemize}

\vspace{3pt}

\item{\textbf{Kibi Development:} \\\vspace{1pt}\textit{'Software development and hardware design'}

\small{Kibi is a product under development in Teseo Srl. It is a solution for domestic elderly care assistance which is composed of: an indoor localization module using BLE technology, gesture recognition module which has the aim to classify different gesture types and an AI module that collects the underlying data and outputs a friendly aggregate. In this context, I am designing the hardware specifications and developing firmware for the wearable devices and anchors which will collect inertial and location data and localize the patient.}}

\item{\textbf{Research Fellowship:} \\\vspace{1pt}\textit{'Software design for controlling aerial vehicles undergoing cooperative tasks with ground/marine robots'}

\small{ My work was conducted in the context of the Italian project MAREA, a consortium of universities and companies working on robots cooperation and management under search and rescue scenarios. I developed a software, written in C++ under Linux environment, for managing general flight missions as well as performing automatic landing on a floating platform. During my work as researcher I supervised master students for group projects and co-supervised 2 bachelor thesis.}}

\vspace{3pt}

\item{\textbf{Masters Thesis:} \\\vspace{1pt}\textit{'Algorithms for controlling and tracking UAVs in indoor scenarios'}

\small{Integration of an Optitrack motion capture system and development of a Qt ground station enabling the robot to perform lists of tasks in an autonomous way. Design and testing of an algorithm for automatic landing on moving targets.}}

\vspace{6pt}

\item{\textbf{University Experience:}

\vspace{3pt}

\small{Hands-on and theoretical experience in: path planning, AI, linear / non-linear analysis of dynamical systems, control and state estimation, optimization algorithms, embedded systems, mobile robots and robot modeling, programming of industrial manipulators.}}

\end{itemize}

\vspace{5pt}

\section{Publications}

\vspace{5pt}

\subsection{Conferences}
\vspace{2pt}
\begin{itemize}
\item \textbf{Nisticò A.}, Baglietto M., Casalino G., Simetti E., Sperindè A.,   \textit{"Marea project: UAV Landing procedure on a
moving and floating platform"}: Oceans ’17 MTS/IEEE , September 18, 2017, Anchorage, USA.
\end{itemize}
\vspace{5pt}

\section{Technical and Personal Skills}

\vspace{6pt}

\begin{itemize}

\item \textbf{Programming Languages:} C , C++, Solidity (proficient), Kotlin; Python, C-Sharp, JavaScript (intermediate)
\vspace{6pt}

\item \textbf{Tools and frameworks:} ROS, LCM middleware, Matlab and Simulink, Git and Travis CI,  Truffle, Android app development (Advanced); PX4 Autopilot Firmware, Embedded STM32 programming with CubeMX configurator,FreeRTOS, CMake, make and gcc toochain, Vagrant (Intermediate); OpenCV, Qt5, NodeJS (Basic).\\
Very good knowledge of Linux environment as a development tool. 
\vspace{6pt}

\item \textbf{General Business Skills:} Good presentation skills and problem solving, Works well in a team.

\vspace{6pt}

\item \textbf{Other:} Experience with embedded systems, Qt libraries, and MavLink protocol. Basic experience with Unity3d game engine. Can write well organized and structured reports.

\vspace{6pt}

\item \textbf{Languages:} English (Fluent), French (Basic), Italian (Native).

\end{itemize}

\vspace{12pt}
\section{Links}
\vspace{10pt}
\begin{itemize}
    \item Certificates:
        \begin{itemize}
            \item FreeRTOS Online training (short link): \url{https://bit.ly/3ppdcOF}
            \item Ethereum (short link): \url{https://bit.ly/2ExiR1h}
            \item HyperLedger Fabric (short link): \url{https://bit.ly/2R4wjk4}
        \end{itemize}
    \item Personal Github: \url{https://github.com/andrea-nisti}
    \item EMAROLab Github: \url{https://github.com/EmaroLab}
    \item Projects in EMAROLab repository:
    \begin{itemize}
        \item Mission management and task execution for UAVs: \url{https://github.com/EmaroLab/mocap2mav}
    \end{itemize}
\end{itemize}

\vspace{12pt}

\section{References}

\vspace{10pt}

\cvdoublecolumn{\cvreference{Fulvio Mastrogiovanni}
    {Department of Informatics, Robotics, \newline Bioengineering and System Engineering}
    {University of Genoa}
    {Via All'Opera Pia, 13}
    {16145 Genoa, Italy}
    {fulvio.mastrogiovanni@unige.it}
    {(+39) 010353-2324}\\
    }
    {\cvreference{Marco Baglietto}
    {Department of Informatics, Robotics, \newline Bioengineering and System Engineering}
    {University of Genoa}
    {Via All'Opera Pia, 13}
    {16145 Genoa, Italy}
    {marco.baglietto@unige.it}
    {(+39) 010353-6548}
    }

% \section{Interests and hobbies}

% \vspace{6pt}

% \begin{itemize}

% \item Traveling

% \vspace{6pt}

% \item Maker culture

% \vspace{6pt}

% \item Homebrewing

% \vspace{6pt}

% \end{itemize}

% \vspace{6pt}

% Publications from a BibTeX file without multibib
%  for numerical labels: \renewcommand{\bibliographyitemlabel}{\@biblabel{\arabic{enumiv}}}% CONSIDER MERGING WITH PREAMBLE PART
%  to redefine the heading string ("Publications"): \renewcommand{\refname}{Articles}
\nocite{*}
\bibliographystyle{plain}
\bibliography{publications}                        % 'publications' is the name of a BibTeX file

% Publications from a BibTeX file using the multibib package
%\section{Publications}
%\nocitebook{book1,book2}
%\bibliographystylebook{plain}
%\bibliographybook{publications}                   % 'publications' is the name of a BibTeX file
%\nocitemisc{misc1,misc2,misc3}
%\bibliographystylemisc{plain}
%\bibliographymisc{publications}                   % 'publications' is the name of a BibTeX file

%-----       letter       ---------------------------------------------------------

\end{document}


%% end of file `template.tex'.